\documentclass[onecolumn,draftclsnofoot, 10pt, compsoc]{IEEEtran}

\usepackage{graphicx}
\usepackage[section]{placeins}
\usepackage{caption}

\usepackage{amssymb}                                         
\usepackage{amsmath}                                         
\usepackage{amsthm}                                

\usepackage{alltt}                                           
\usepackage{float}
\usepackage{color}
\usepackage{url}

\usepackage{balance}
\usepackage[TABBOTCAP, tight]{subfigure}
\usepackage{enumitem}
\usepackage{pstricks, pst-node}
\usepackage{url}
\usepackage{setspace}
\usepackage{xspace}
\usepackage{array}
\usepackage{hyperref}
\usepackage{etoolbox}
\AtBeginEnvironment{quote}{\singlespacing\vspace{-\topsep}\small}

%input{pygments.tex}

\usepackage{geometry}
\geometry{left=0.75in,right=0.75in,top=0.75in,bottom=0.75in}
\parindent = 0.0 in
\parskip = 0.1 in

\usepackage{fancyvrb}
\usepackage{textcomp}

\def \ParSpace{\vspace{.75em}}
\def \GroupNumber{17\xspace}
\def \Class{Operating Systems II\xspace}
\def \School{Oregon State University\xspace}
\def \Professor{Kevin McGrath\xspace}

\newcommand{\cred}[1]{{\color{red}#1}}
\newcommand{\cblue}[1]{{\color{blue}#1}}

\newcommand{\NameSigPair}[1]{
		\par
		\makebox[2.75in][r]{#1} \hfil 	\makebox[3.25in]{\makebox[2.25in]{\hrulefill} \hfill			
		\makebox[.75in]{\hrulefill}}
		\par\vspace{-12pt} \textit{
			\tiny\noindent
			\makebox[2.75in]{} \hfil		
			\makebox[3.25in]{
				\makebox[2.25in][r]{Signature} \hfill	\makebox[.75in][r]{Date}
			}
		}
}



%%%%%%%%%%%%%%%%%%%%%%%%%%%%%%%%%%%%%%%
\begin{document}
\begin{titlepage}
    \pagenumbering{gobble}
    \begin{singlespace}
    	\includegraphics[height=4cm]{coe.eps}
        \hfill  
        \par\vspace{.2in}
        \centering
        \scshape{
            \vspace{.5in}
            \textbf{\Huge\Class}\par
            \large{
            	\today \\Spring Term
        	}
            \vfill
            {\large Prepared for}\par
            \huge \School\par
            \vspace{5pt}
            {\Large{\Professor}\par}
            {\large Prepared by }\par
           % Group\GroupNumber\par
            \vspace{5pt}
            {\Large
                {Austin Row}\par
                {Lazar Sharipoff}\par
                {Benjamin Richards}\par
            }
            \vspace{20pt}
        }
        \begin{abstract}
			This document describes the steps used by group \GroupNumber to complete Project 2 in which we implemented a LOOK IO Scheduler for the Linux kernel for \Class. 
			It also answers the questions given in the assignment description regarding the assignment itself.
        \end{abstract}     
    \end{singlespace}
\end{titlepage}
\newpage
\pagenumbering{arabic}
\tableofcontents
% 7. uncomment this (if applicable). Consider adding a page break.
%\listoffigures
%\listoftables
\clearpage


\section{Steps To Setup Environment}
	Note: This assumes that the files \textbf{yocto\_config}, \textbf{look\_scheduler.patch}, and \textbf{sstf-iosched.c} which were included in this submission have been copied to your home directory on os2.engr.oregonstate.edu.
	\begin{enumerate}
		\item
			SSH on to the OS2 server: \textit{ssh os2.engr.oregonstate.edu. }
		\item
			Clone the yocto repo: \textit{git clone git://git.yoctoproject.org/linux-yocto}
		\item 
			Change to the correct yocto version with \textit{cd linux-yocto} followed by \textit{checkout v3.19.2}
		\item 
			\textit{cp \texttildelow/yocto\_config ./.config}
		\item 
			\textit{cp \texttildelow/look\_scheduler.patch ./}
		\item 
			\textit{cp \texttildelow/sstf-iosched.c ./block/}
		\item 
			\textit{git apply look\_scheduler.patch}
	\end{enumerate}


\section{Steps To Build and Run Kernel}
	\begin{enumerate}
		\item
			Split the terminal into two windows using tmux.
			We will refer to these windows as T1 and T2. 
		\item
			T1: \textit{source /scratch/files/environment-setup-i586-poky-linux} \\
			It will appear that nothing happened, but it actually ran commands to configure the shell so that it’s ready to build the kernel. 
		\item
			T1: \textit{make -j4 all} \\
			This is the command that actually builds the kernel and generates the kernel image that will used to boot the operating system within the VM. 
			This command will take several minutes. 
			Wait until it is done before going on to the next step.
		\item
			T1: 
			\textit{qemu-system-i386 -gdb tcp::5517 -S -nographic -kernel ./arch/x86/boot/bzImage -drive file=/scratch/spring2018/17/core-image-lsb-sdk-qemux86.ext4 -enable-kvm -net none -usb -localtime --no-reboot --append "root=/dev/hda rw console=ttyS0 elevator=look"} \\
			The terminal should hang now and do nothing. 
			You have booted the VM using the bzImage file that you generated in the previous step. 
			The VM’s CPU is halted which is why nothing is happening. 
			Proceed to the next step.	
		\item
			T2: \textit{gdb -tui}
		\item
			T2: \textit{target remote :5517} \\
			This attaches the current GDB process to the process which is running on port 5517 which is the process running the halted VM.
		\item
			T2 : \textit{continue} \\
			Now the VM in T1 will no longer be hanging. 
			Switch back to T1 and wait for it to ask you to login.
		\item
			T1: \textit{root} \\
			You are now within the VM logged in as the root user and using a kernel that utilizes a LOOK IO Scheduler. 
	\end{enumerate}

\clearpage
\section{LOOK Algorithm Design}
	The queue kept by the scheduler can be thought of as two adjacent segments sorted in different directions. 
	If the scheduler is moving in the UP direction, the first segment will be sorted in ascending order and it will be followed either by nothing or a segment with elements sorted in descending order and vice versa.
	With this in mind, the algorithm for inserting into the queue comes down to determining which segment to put the new element in and then where in the segment the element should be placed to maintain the segment's ordering.
	This gives the following algorithm: \\

	Note: iterator.position refers to the position on the disk of the block for the request that iterator points to.

\begin{Verbatim}[commandchars=+\[\]]
+underline[AddRequest(request, queue)]
if empty(queue):
	queue.append(request)
else:
	iterator = queue.firstElement
	iteratorNext = iterator.next

	if (iterator.position < request.position AND currectDirection == DOWN)
		OR
	   (iterator.position > request.position AND currectDirection == UP):

		//New request does not belong in first segment, 
		//iterate to right before second segment.
		while (currectDirection == UP 
			   AND 
			   iterator.position < iteratorNext.position)
			  OR
			  (currectDirection == DOWN 
			   AND 
			   iterator.position > iteratorNext.position):
				  iterator = iteratorNext
				  iteratorNext = iterator.next

	//At this point, iterator will be on the first element of 
	//the first segment if the new request belongs there
	//or it will be on the element before the first of the second 
	//segment if that's where it belongs.


	//While we haven't reached the last element and the new request 
	//belongs after iteratorNext in the segment
	while iterator != queue.lastElement
		  AND
		  (iterator.position < iteratorNext.position) == 
		  (iteratorNext.position < request.position)
			iterator = iteratorNext
			iteratorNext = iterator.next

	//At this point iterator points to the element after 
	//which the new request should be placed
	Add request to queue after iterator
\end{Verbatim}


\section{Questions About Assignment}
	\begin{enumerate}
		\item What do you think the main point of this assignment is? \\
			The main point of this assignment is to learn what IO Schedulers are, what responsibilities they have, and in what ways they can differ.
			These topics are explored via a hands on approach where we are forced to learn if we want to create a functioning scheduler of our own.
		\item How did you personally approach the problem? \\
			The first step that we took in completing the assignment was to learn about the different responsibilities of the IO scheduler in the Linux kernel.
			We had to first know what functionality we were looking to implement before we could begin to think about how to implement it.
			Next we looked at how the other existing schedulers were implemented.
			This helped us understand what common components every scheduler should have as well as what implementation patterns should be followed while implementing ours.
			After looking at each scheduler, we copied the source for the noop scheduler to use as a basis for the look scheduler. 
			At this point we knew that we needed to change the data that the scheduler was keeping track of (e.g. the look scheduler needs to know what direction it is moving in) as well as how it handled adding and merging requests.
			We began by changing how the scheduler added requests to its queue since that is the core of what defines a scheduler.
			After changing the implementation of the function to add requests to one that used the LOOK algorithm, we implemented front merges for merging requests.
		\item How did you ensure your solution was correct? \\
			Kernel print statements were put after each addition or dispatch of a request which displayed the positions of each of the requests in the scheduler's queue as well as after each merge that occured.
			This allowed us to look at the kernel message log and determine that the queue was changing as one would expect while using a scheduler based on the LOOK algorithm.
		\item What did you learn? \\
			Throughout the course of this assignment we learned:
			\begin{itemize}
				\item
					That schedulers are responsible for deciding in which order requests to disk should be added to the dispatch queue.
				\item
					That schedulers are responsible for implementing the merging of requests that are in adjacent blocks on disk.
				\item 
					That the primary difference between schedulers is how they determine the order in which requests should be added to the dispatch queue.
				\item
					How to use the API for kernel lists.
				\item
					How to change what scheduler the kernel uses.
				\item
					How to view kernel log messages and the different priority levels for these messages.
				\item
					How to change what log message levels are printed to the console.
			\end{itemize}
		\item How should the TA evaluate your work? \\
			Running \textit{dmesg $|$ grep "***LOOK\_SCHED"} will show all of the kernel messages printed during the execution of the scheduler.
			However, there can be a lot of these, so it is recommended that you use some of the following instead:
			\begin{itemize}
				\item
					Show the block positions of the requests queued by the scheduler when there are at least 4 requests in the queue: \textit{dmesg $|$ grep "LOOK\_LONG"}
				\item 
					Show the block positions of the requests queued by the scheduler when there are some that are queued in ascending order: \textit{dmesg $|$ grep "LOOK\_ASCENDING"}
				\item 
					Show the block positions of the requests queued by the scheduler when there are some that are queued in descending order: \textit{dmesg $|$ grep "LOOK\_DESCENDING"}
				\item 
					Show the block positions of the requests queued by the scheduler when there are some that are queued in both ascending and descending order: \textit{dmesg $|$ grep "LOOK\_ASCENDING LOOK\_DESCENDING"}
				\item
					Show messages that indicate a front merge was found: \textit{dmesg $|$ grep "Found front merge"}
				\item 
					Show when the scan direction switches to UP: \textit{dmesg $|$ grep "Scan direction changed to UP"}
				\item 
					Show when the scan direction switches to DOWN: \textit{dmesg $|$ grep "Scan direction changed to DOWN"}
				\item
					Show when a request is added to the queue: \textit{dmesg $|$ grep "Adding request for position"}
			\end{itemize}
			Correctness can be discerned by the fact that the request positions stored in the queue are always in ascending or descending order with the direction changing at most once within any particular instance of the queue.
	\end{enumerate}


\section{Version Control Log}
\begin{tabular}{|p{5cm}|p{5cm}|p{5cm}|}
\hline
\textbf{Commit} & \textbf{Author} & \textbf{Description}\\
\hline
\textcolor{blue}{\underline{\href[pdfnewwindow=true]{https://github.com/AustinRow1/OperatingSystems2/commit/fad3a1690038d1785a07071dea724c7fa74d0846}{fad3a16}}} & AustinRow1 & Implemented look scheduler for yocto.\\\hline
\textcolor{blue}{\underline{\href[pdfnewwindow=true]{https://github.com/AustinRow1/OperatingSystems2/commit/0e9906640572cc0be52dbe039061e91fa5622da1}{0e99066}}} & AustinRow1 & Created patch file, deleted unneccessary files.\\\hline
\textcolor{blue}{\underline{\href[pdfnewwindow=true]{https://github.com/AustinRow1/OperatingSystems2/commit/a98b7c18aa474b141b50f364620e2b76a3790157}{a98b7c1}}} & AustinRow1 & Started writeup for look scheduler assignment and added look scheduler source file to assignment directory.\\\hline
\textcolor{blue}{\underline{\href[pdfnewwindow=true]{https://github.com/AustinRow1/OperatingSystems2/commit/297ac1df3960a66e565b1b62f6dfe6fcc8adce48}{297ac1d}}} & AustinRow1 & Completed general assignment questions section of look\_scheduler.tex\\\hline\end{tabular}


\section{Work Log}
	\begin{center}
		\begin{tabular}{ |p{2cm}|p{12cm}| }
			\hline
			5/3/2018 & Did the entire assignment including learning about io schedulers \cite{io-schedulers}, learning what the LOOK algorithm was \cite{look-algorithm}, and implementing it with the help of the kernel API source \cite{kernel-api-source} and the kernel linked list documentation \cite{kernel-list-doc}. \\
			\hline
		\end{tabular}
	\end{center}


\bibdata{}
\bibliographystyle{IEEEtran}
\bibliography{references}

\end{document}
