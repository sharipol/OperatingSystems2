\documentclass[onecolumn,draftclsnofoot, 10pt, compsoc]{IEEEtran}

\usepackage{graphicx}
\usepackage[section]{placeins}
\usepackage{caption}

\usepackage{amssymb}                                         
\usepackage{amsmath}                                         
\usepackage{amsthm}                                

\usepackage{alltt}                                           
\usepackage{float}
\usepackage{color}
\usepackage{url}

\usepackage{balance}
\usepackage[TABBOTCAP, tight]{subfigure}
\usepackage{enumitem}
\usepackage{pstricks, pst-node}
\usepackage{url}
\usepackage{setspace}
\usepackage{xspace}
\usepackage{array}
\usepackage{etoolbox}
\AtBeginEnvironment{quote}{\singlespacing\vspace{-\topsep}\small}

%input{pygments.tex}

\usepackage{geometry}
\geometry{left=0.75in,right=0.75in,top=0.75in,bottom=0.75in}
\parindent = 0.0 in
\parskip = 0.1 in


\def \ParSpace{\vspace{.75em}}
\def \GroupNumber{17\xspace}
\def \Class{Operating Systems II\xspace}
\def \School{Oregon State University\xspace}
\def \Professor{Kevin McGrath\xspace}

\newcommand{\cred}[1]{{\color{red}#1}}
\newcommand{\cblue}[1]{{\color{blue}#1}}

\newcommand{\NameSigPair}[1]{
		\par
		\makebox[2.75in][r]{#1} \hfil 	\makebox[3.25in]{\makebox[2.25in]{\hrulefill} \hfill			
		\makebox[.75in]{\hrulefill}}
		\par\vspace{-12pt} \textit{
			\tiny\noindent
			\makebox[2.75in]{} \hfil		
			\makebox[3.25in]{
				\makebox[2.25in][r]{Signature} \hfill	\makebox[.75in][r]{Date}
			}
		}
}



%%%%%%%%%%%%%%%%%%%%%%%%%%%%%%%%%%%%%%%
\begin{document}
\begin{titlepage}
    \pagenumbering{gobble}
    \begin{singlespace}
    	\includegraphics[height=4cm]{coe.eps}
        \hfill  
        \par\vspace{.2in}
        \centering
        \scshape{
            \vspace{.5in}
            \textbf{\Huge\Class}\par
            \large{
            	\today \\Spring Term
        	}
            \vfill
            {\large Prepared for}\par
            \huge \School\par
            \vspace{5pt}
            {\Large{\Professor}\par}
            {\large Prepared by }\par
           % Group\GroupNumber\par
            \vspace{5pt}
            {\Large
                {Austin Row}\par
                {Lazar Sharipoff}\par
                {Benjamin Richards}\par
            }
            \vspace{20pt}
        }
        \begin{abstract}
			This document describes the steps used by group \GroupNumber to complete Project 1 for \Class. 
			It also answers the questions given in the assignment description regarding the assignment itself.
        \end{abstract}     
    \end{singlespace}
\end{titlepage}
\newpage
\pagenumbering{arabic}
\tableofcontents
% 7. uncomment this (if applicable). Consider adding a page break.
%\listoffigures
%\listoftables
\clearpage

\section{Steps To Setup Environment}
	\begin{enumerate}
		\item
			SSH into os2.engr.oregonstate.edu and change directories to \textbf{/scratch/spring2018/}.
		\item
			Run \textbf{mkdir \GroupNumber} and change directories to \textbf{/scratch/spring2018/\GroupNumber}.
		\item
			Clone the linux-yocto repository into this group directory by running \textbf{git clone http://git.yoctoproject.org/cgit.cgi/linux-yocto/}.
		\item 
			Copy \textbf{bzImage-qemux86.bin}, \textbf{core-image-lsb-sdk-qemux86.ext4}, and \textbf{environment-setup-i586-poky-linux} from /scratch/files/ to /scratch/spring2018/17/.
	\end{enumerate}

\section{Steps To Build and Run Kernel}
	\begin{enumerate}
		\item
			Open two terminal windows and ssh into os2.engr.oregonstate.edu on both. 
			We will refer to these windows as T1 and T2. 
			For both windows, navigate to the group directory, \textbf{/scratch/spring2018/17/}.	
		\item
			In T1, run \textbf{source ./environment-setup-i586-poky-linux}. 
			It will appear that nothing happened, but it actually ran commands to configure the shell so that it’s ready to build the kernel. 
		\item
			In T1, change directories into linux-yocto and run \textbf{git checkout v3.19.2} to switch to Yocto version 3.19.2.
			Run \textbf{make -j4 all}. 
			This is the command that actually builds the kernel and generates the kernel image that will used to boot the operating system within the VM. 
			This command will take several minutes. 
			Wait until the command is done before going on to the next step.
		\item
			In T1, run 
			\textbf{qemu-system-i386 -gdb tcp::5517 -S -nographic -kernel ./arch/x86/boot/bzImage -drive file=../core-image-lsb-sdk-qemux86.ext4,if=virtio -enable-kvm -net none -usb -localtime --no-reboot --append "root=/dev/vda rw console=ttyS0 debug"}. 
			The terminal should hang now and do nothing. 
			You have booted the VM using the bzImage file that you generated in the previous step. 
			The VM’s CPU is halted which is why nothing is happening. 
			Proceed to the next step.	
		\item
			In T2 (note we have switched terminal windows), run \textbf{gdb -tui} which will launch the GDB debugger with terminal user interface which is just a prettier way to interact with GDB.
		\item
			In T2 on the GDB command line, run \textbf{target remote :5517}. 
			This attaches the current GDB process to the process which is running on port 5517 which is the process running the halted VM.
		\item
			Now to make it so that the VM is no longer halted, in T2 on the GDB command line, run \textbf{continue}. 
			Now the VM in T1 will no longer be hanging. 
			Switch back to T1 and wait for it to ask you to login.
		\item
			In T1 once prompted to login, type \textbf{root} and press Enter. 
			You are now within the VM logged in as the root user and using an operating system that utilizes the kernel which you built earlier. 
	\end{enumerate}

\section{Qemu Command Explanation}
	\textit{qemu-system-i386 -gdb tcp::5517 -S -nographic -kernel ./arch/x86/boot/bzImage -drive file=../core-image-lsb-sdk-qemux86.ext4,if=virtio -enable-kvm -net none -usb -localtime --no-reboot --append "root=/dev/vda rw console=ttyS0 debug"}. 
	\begin{itemize}
		\item \textbf{-gdb tcp::5517} \\
		Wait for a gdb connection on TCP port 5517.
		\item \textbf{-S} \\
		Do not start CPU at startup.	
		\item \textbf{-nographic} \\
		Disable graphical output so as to provide only a command line interface.
		\item \textbf{-kernel ./arch/x86/boot/bzImage} \\
		Use ./arch/x86/boot/bzImage as the kernel image.	
		\item \textbf{-drive file=../core-image-lsb-sdk-qemux86.ext4,if=virtio} \\
		Create a new drive using the disk image ../core-image-lsb-sdk-qemux86.ext4 and associate it with a virtio interface.
		\item \textbf{-enable-kvm} \\
		Enable support for KVM full virtualization.
		\item \textbf{-net none} \\
		Do not configure any network devices for the VM.
		\item \textbf{-usb} \\
		Enable the USB driver.	
		\item \textbf{-localtime} \\
		Use localtime to set the real-time clock.
		\item \textbf{--no-reboot} \\ 
		Do not allow reboot, exit instead.
		\item \textbf{--append "root=/dev/vda/rw console=ttyS0 debug"} \\
		Add "root=/dev/vda/rw console=ttyS0 debug" to command line configuration in VM.
	\end{itemize}

\section{Questions About Assignment}
	\begin{enumerate}
		\item What do you think the main point of this assignment is? \\
			The main point of this assignment is to make us familiar with the process of building and running the kernel in a VM.
			It also gets us to understand exactly what is being done in each of the steps that we perform to run the kernel which will help us later when we want to modify what we are doing with the kernel.
		\item How did you personally approach the problem? \\
			Most of the necessary information regarding what to do was included in the assignment description.
			However, some of the instructions given referenced things that nobody in the group understood which made it difficult to correctly use the instructions at first.
			Therefore, the first step was to look at each step in the instructions separately and to learn exactly what that step was doing.
			In doing this, we were able to better understand how the commands given in the instructions fit together and what role each command played in the overall process of running the kernel.
			Once we had this understanding, we were able to execute the instructions properly and complete the assignment.
		\item How did you ensure your solution was correct? \\
			TODO
		\item What did you learn? \\
			Throughout the course of this assignment we learned:
			\begin{itemize}
				\item
					How to build the kernel in the linux-yocto repo.
				\item
					How to use the kernel image that we built to run an operating system in a VM using QEMU.
				\item 
					That kernel and disk images are files that contain all of the files needed to reproduce a kernel or drive respectively in the specific state captured in the image file.
				\item 
					What the qemu-system-i386 command does along with how the various options we give it affect what it does.
				\item 
					How to change kernel configurations using \textit{make menuconfig}.		
			\end{itemize}
	\end{enumerate}

\section{Version Control Log}
	TODO
 
\section{Work Log}
\begin{center}
	\begin{tabular}{ |p{2cm}|p{12cm}| }
		\hline
		4/4/2018 & Created group directory and GitHub repo and copied files from \textit{/scratch/files/} into it.
		           Also cloned yocto-3.19.2 source into the directory. \\
		\hline
		4/11/2018 & Deleted yocto-3.19.2 repo from group directory and cloned linux-yocto into it.
		            Completed steps to get VM running with kernel from yocto v3.19.2. \\
		\hline
		4/12/2018 & Started writeup. \\
		\hline
		4/13/2018 & Completed writeup. \\
		\hline
	\end{tabular}
\end{center}
 
%\bibliographystyle{IEEEtran}
%\bibliography{references}

\end{document}
